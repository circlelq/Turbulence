%! TEX root = /Users/circle/Documents/博二上/湍流/Turbulence/HW3/homework/homework.tex
\documentclass[12pt,a4]{ctexart}
\input{/Users/circle/Documents/博二上/homework/setting.tex}
\setcounter{secnumdepth}{2}
\usepackage{bm}
\usepackage{autobreak}
\usepackage{amsmath}
\setlength{\parindent}{2em}
\graphicspath{{../}}


%pdf 文件设置
\hypersetup{
   pdfauthor={袁磊祺},
   pdftitle={湍流3}
}

\title{
   \vspace{-1in}
   \usefont{OT1}{bch}{b}{n}
   \normalfont \normalsize \textsc{\LARGE Peking University}\\[1cm] % Name of your university/college \\ [25pt]
   \horrule{0.5pt} \\[0.5cm]
   \huge \bfseries{湍流3} \\
   \horrule{2pt} \\[0.5cm]
}
\author{
   \normalfont									\normalsize
   College of Engineering \quad 2001111690  \quad 袁磊祺\\	\normalsize
   \today
}
\date{}

\begin{document}

\input{setc.tex}

\maketitle

2021 年 11 月 27 日前交电子版.

代码等作业内容可在 \texttt{\href{https://github.com/circlelq/Turbulence}{https://github.com/circlelq/Turbulence}} 查看。


\section{1}

\textsf{\hspace{-2em}\sf  \textbf{解:}}

对于归一化的一维光滑平稳高斯过程 $X(t)$, 即满足 $\langle X\rangle=0,\left\langle X^{2}\right\rangle=1$, 记 $\dot{X}=$ $\dif X / \dif t$, 证明如下两个条件平均关系: $\langle\ddot{X} \mid X=x\rangle=-\left\langle\dot{X}^{2}\right\rangle x ;\, \left\langle\dot{X}^{2} \mid X=x\right\rangle=\left\langle\dot{X}^{2}\right\rangle$ 。 提示: 应该首先说明或者证明 $\dot{X}(t)$ 和 $\ddot{X}(t)$ 也是高斯过程。

\section{2}
对于不可压均匀各向同性湍流, 试给出两空间点的涡量速度关联张量的最简 表达式。


\section{3}
对于不可压均匀各向同性湍流, 根据不可压条件 (连续性方程) 和湍流统计量 与构型(configuration)的方向无关的特点, 通过标架旋转证明一点的速度梯 度满足统计关系:
\begin{equation}
   \left\langle\left(\frac{\partial u_{1}}{\partial x_{2}}\right)^{2}\right\rangle=2\left\langle\left(\frac{\partial u_{1}}{\partial x_{1}}\right)^{2}\right\rangle.
\end{equation}
提示: 可参阅 G.I. Taylor 在 1935 年发表的关于均匀向同性湍流的有关论文。


为了书写方便,记
\begin{equation}
   u = u_1,\, v = u_2,\, w = u_3,\, x = x_1,\, y = x_2,\, z = x_3.
\end{equation}

由不可压有
\begin{equation}
   \frac{\partial u}{\partial x} + \frac{\partial v}{\partial y} + \frac{\partial w}{\partial z} = 0,
\end{equation}
即
\begin{equation}
   \frac{\partial u}{\partial x} = - \frac{\partial v}{\partial y} - \frac{\partial w}{\partial z},
\end{equation}
两边平方得
\begin{equation}
   \begin{aligned}
	  \left( \frac{\partial u}{\partial x} \right)^2 & =  \left( \frac{\partial v}{\partial y} + \frac{\partial w}{\partial z} \right)^2                                                                                   \\
													 & = \left( \frac{\partial v}{\partial y}  \right)^2 + \left( \frac{\partial w}{\partial z}  \right)^2 + 2\frac{\partial v}{\partial y} \frac{\partial w}{\partial z}.
   \end{aligned}
\end{equation}
取系综平均可得
\begin{equation}
   \left< \left( \frac{\partial u}{\partial x} \right)^2 \right> = \left<\left( \frac{\partial v}{\partial y}  \right)^2 \right> + \left<\left( \frac{\partial w}{\partial z}  \right)^2 \right> + \left<2\frac{\partial v}{\partial y} \frac{\partial w}{\partial z}
   \right>
   \label{eq:31}
\end{equation}
由各向同性湍流可得\cite{10.2307/96557}
\begin{equation}
   \left< \left( \frac{\partial u}{\partial x} \right)^2 \right> =\left< \left( \frac{\partial v}{\partial y} \right)^2 \right> =\left< \left( \frac{\partial w}{\partial z} \right)^2 \right>
\end{equation}
所以\cref{eq:31}化为
\begin{equation}
   \left< \left( \frac{\partial u}{\partial x} \right)^2 \right> = -2 \left<\frac{\partial v}{\partial y} \frac{\partial w}{\partial z}
   \right>
\end{equation}
对坐标进行$45^{\circ}$可得
\begin{equation}
   \begin{cases}
	  \sqrt{2} x' & = x + y  \\
	  \sqrt{2} y' & = -x + y \\
	  z' = z
   \end{cases}
   \label{eq:32}
\end{equation}
\begin{equation}
   \begin{cases}
	  \sqrt{2} u' & = u + v  \\
	  \sqrt{2} v' & = -u + v \\
	  v' = v
   \end{cases}
\end{equation}
Hence
\begin{equation}
   \begin{cases}
	  \frac{\partial u}{\partial x} & =\frac{1}{2}\left(\frac{\partial u^{\prime}}{\partial x^{\prime}}-\frac{\partial v^{\prime}}{\partial x^{\prime}}-\frac{\partial u^{\prime}}{\partial y^{\prime}}+\frac{\partial v^{\prime}}{\partial y^{\prime}}\right) \\
	  \frac{\partial v}{\partial x} & =\frac{1}{2}\left(\frac{\partial u^{\prime}}{\partial x^{\prime}}+\frac{\partial v^{\prime}}{\partial x^{\prime}}-\frac{\partial u^{\prime}}{\partial y^{\prime}}-\frac{\partial u^{\prime}}{\partial y^{\prime}}\right) \\
	  \frac{\partial w}{\partial x} & =\frac{1}{\sqrt{2}}\left(\frac{\partial w^{\prime}}{\partial x^{\prime}}-\frac{\partial w^{\prime}}{\partial y^{\prime}}\right)                                                                                          \\
   \end{cases}
\end{equation}
\begin{equation}
   \begin{cases}
	  \frac{\partial u}{\partial y} & =\frac{1}{2}\left(\frac{\partial u^{\prime}}{\partial x^{\prime}}-\frac{\partial v^{\prime}}{\partial x^{\prime}}+\frac{\partial u^{\prime}}{\partial y^{\prime}}-\frac{\partial v^{\prime}}{\partial y^{\prime}}\right) \\
	  \frac{\partial v}{\partial y} & =\frac{1}{2}\left(\frac{\partial u^{\prime}}{\partial x^{\prime}}+\frac{\partial v^{\prime}}{\partial x^{\prime}}+\frac{\partial u^{\prime}}{\partial y^{\prime}}+\frac{\partial v^{\prime}}{\partial y^{\prime}}\right) \\
	  \frac{\partial w}{\partial y} & =\frac{1}{\sqrt{2}}\left(\frac{\partial w^{\prime}}{\partial x^{\prime}}+\frac{\partial w^{\prime}}{\partial y^{\prime}}\right)                                                                                          \\
   \end{cases}
\end{equation}
\begin{equation}
   \begin{cases}
	  \frac{\partial u}{\partial z} & =\frac{1}{\sqrt{2}}\left(\frac{\partial u^{\prime}}{\partial z^{\prime}}-\frac{\partial v^{\prime}}{\partial z^{\prime}}\right) \\
	  \frac{\partial v}{\partial z} & =\frac{1}{\sqrt{2}}\left(\frac{\partial u^{\prime}}{\partial z^{\prime}}+\frac{\partial v^{\prime}}{\partial z^{\prime}}\right) \\
	  \frac{\partial w}{\partial z} & =\frac{\partial w^{\prime}}{\partial z^{\prime}}
   \end{cases}
\end{equation}

\begin{table}[htpb]
   \centering
   \caption{记号说明。}
   \label{tab:31}
   \begin{tabular}{ccccccccccc}
	  \toprule
	  $\overline{\left(\frac{\partial u}{\partial x}\right)^{2}}$
	   & $\overline{ \frac{\partial u}{\partial x} \frac{\partial u}{\partial y} }$
	   & $\overline{{\left(\frac{\partial u}{\partial y}\right)}^{2} }$
	   & $\overline{  \frac{\partial u}{\partial y} \frac{\partial u}{\partial z} }$
	   & $\overline{ \frac{\partial u}{\partial x} \frac{\partial v}{\partial x} }$
	   & $ \overline{\frac{\partial u}{\partial x} \frac{\partial v}{\partial y}} $
	   & $ \overline{\frac{\partial u}{\partial x} \frac{\partial v}{\partial z}}$
	   & $\overline{ \frac{\partial u}{\partial y} \frac{\partial v}{\partial x} }$
	   & $\overline{ \frac{\partial u}{\partial y} \frac{\partial v}{\partial z} }$
	   & $\overline{ \frac{\partial u}{\partial z} \frac{\partial v}{\partial z}}$   \\
	   \midrule
	   $a_{1} $
	   & $ a_{2} $
	   & $ a_{3} $
	   & $ a_{4} $
	   & $ a_{5} $
	   & $ a_{6} $
	   & $ a_{7} $
	   & $ a_{8} $
	   & $ a_{9} $
	   & $ a_{10}$\\
	   \bottomrule
   \end{tabular}
\end{table}
对\cref{eq:32}求平方,并利用各向同性可得以下等式
\begin{equation}
   a_4 = a_7 = a_2 = a_5 = a_{10} = a_9 = 0.
\end{equation}
\begin{equation}
   a_1 = - 2a_6,
\end{equation}
\begin{equation}
   a_1 - a_3 - a_6 - a_8 = 0,
\end{equation}
\begin{equation}
   a_1 - a_3 - a_6 - a_8 = 0,
\end{equation}
\begin{equation}
   a_1 + 2a_8 = 0,
\end{equation}
\begin{equation}
   a_1 = - 2a_6
\end{equation}
\begin{equation}
   a_1 = \frac{1}{2} a_3.
\end{equation}
\qed


\section{4}

将不可压均匀各向同性湍流中两点纵向速度关联函数与一维能谱之间的傅立 叶积分变换关系
\begin{equation}
   \left\langle u^{2}\right\rangle f(r)=\int_{-\infty}^{\infty} \phi_{1}(k) \me^{\mi k r}\dif k
\end{equation}
代入如下两点纵向速度关联函数与三维能谱之间的积分变换关系
\begin{equation}
   E(k)=\frac{1}{\pi} \int_{-\infty}^{\infty}\left\langle u^{2}\right\rangle f(r)(k r)^{2}\left(\frac{\sin k r}{k r}-\cos k r\right) \dif r,
\end{equation}
直接通过计算推出
\begin{equation}
   E(k)=k^{3} \frac{\dif}{\dif k}\left[\frac{1}{k} \frac{\dif \phi_{1}(k)}{\dif k}\right].
\end{equation}
或者, 将
\begin{equation}
   \left\langle u^{2}\right\rangle f(r)=2 \int_{-\infty}^{\infty} E(k)(k r)^{-2}\left(\frac{\sin k r}{k r}-\cos k r\right)\dif k
   \label{eq:41}
\end{equation}
代入傅立叶逆变换关系
\begin{equation}
   \phi_{1}(k)=\frac{1}{2 \pi} \int_{-\infty}^{\infty}\left\langle u^{2}\right\rangle f(r) \me^{-\mi k r} \dif r,
   \label{eq:42}
\end{equation}
进行积分, 推出
\begin{equation}
   \phi_{1}(k)=\frac{1}{2} \int_{k}^{\infty}\left(1-\frac{k^{2}}{\lambda^{2}}\right) \frac{E(\lambda)}{\lambda} \dif \lambda.
\end{equation}
两个推导过程任选其一完成即可。提示: 第一个推导利用 $\delta$ 函数及其导数的性质; 第二个推导用到由如下积分关系
\begin{equation}
   \frac{1}{2} \int_{0}^{1}\left(1-k^{2}\right) \cos k x \dif k=\frac{1}{x^{2}}\left(\frac{\sin x}{x}-\cos x\right)
   \label{eq:43}
\end{equation}
带来的余弦变换。

将\cref{eq:41}带入\cref{eq:42}可得
\begin{equation}
   \begin{aligned}
	  \phi_{1}(k) & = \frac{1}{2 \pi} \int_{-\infty}^{\infty}\left\langle u^{2}\right\rangle f(r) \me^{-\mi k r} \dif r \\ 
				  & = \frac{1}{2 \pi} \int_{-\infty}^{\infty}2 \int_{-\infty}^{\infty} E(\lambda)(\lambda r)^{-2}\left(\frac{\sin \lambda r}{\lambda r}-\cos \lambda r\right)\dif \lambda \, \me^{-\mi k r} \dif r \\
   \end{aligned}
\end{equation}













\section{5}

用不同于教材中的方法证明教材 (2.3.42) 式:
\begin{equation}
   \left\langle u^{2}\right\rangle^{\frac{3}{2}} k(r)=8 \pi \int_{0}^{\infty} \frac{k^{5} \Gamma(k)}{(k r)^{4}}\left(3 \sin k r-3 k r \cos k r-(k r)^{2} \sin k r\right)\dif k.
\end{equation}

\section{6}

自己从网上或者课题组内部获取实验测量或者直接数值模拟得到的湍流脉动 速度信号的一段长时间的时间序列或者一定样本容量的空间序列。完成下列两 项任务:
\begin{enumerate}
   \item 计算两点纵向的二阶和三阶速度关联系数 $f(r),\, k(r)$ 并画出曲线 (可以用泰勒冻结假设);
   \item 根据 $f(r)$ 计算一维能谱 $\phi_{1}(k)$ 和三维能谱 $E(k)$ (在 假设各向同性的情况下利用教材公式 (2.3.23)), 并画出曲线。
\end{enumerate}





% \nocite{*}

\input{bib.tex}

\end{document}
