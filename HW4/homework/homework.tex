%! TEX root = /Users/circle/Documents/博二上/湍流/Turbulence/HW4/homework/homework.tex
\documentclass[12pt,a4]{ctexart}
\usepackage{natbib}
\usepackage{url}
\usepackage{stmaryrd}
\usepackage{mathrsfs}
\usepackage{amsmath}
\usepackage{graphicx}
\usepackage{parskip}
\usepackage{fancyhdr}
% \usepackage{underscore} % 下划线
\usepackage{commath}%定义d
\usepackage{geometry}
\usepackage{bm}
\usepackage{siunitx}
\usepackage{float}
% \usepackage[skip=-5pt]{subcaption}
% \usepackage{subfig}
\usepackage{subfigure}  %插入多图时用子图显示的宏包
\usepackage{titlesec}
\usepackage{caption}
\usepackage{paralist}
\usepackage{multirow}
\usepackage{booktabs} % To thicken table lines
\usepackage{diagbox}
\usepackage{authblk}
\usepackage{indentfirst}
\usepackage{amsthm}
\usepackage{fontspec}
\usepackage{color}
%\usepackage{txfonts} %设置字体为times new roman
\usepackage{lettrine}
\usepackage{nameref}
%\usepackage[nottoc]{tocbibind}
\usepackage{amssymb}%font
\usepackage{lipsum}%make test words
\usepackage{picinpar}%words around the picture
\usepackage[all]{xy}%draw arrow
\usepackage{asymptote}%draw picture
\usepackage[perpage]{footmisc}%脚注每页清零
\usepackage{esint}
\renewcommand{\proofname}{\indent \sf \bfseries{证明}}
\pagestyle{fancy}
\fancyhf{}
\renewcommand{\headrulewidth}{0pt}
\fancyfoot[C]{\thepage}

\catcode`\。=\active
\catcode`\,=\active
\catcode`\;=\active
\catcode`\:=\active
\newcommand{。}{.}
\newcommand{,}{,}
\newcommand{;}{;}
\newcommand{:}{:}

\geometry{bottom=3cm,left=3cm,right=3cm,a4paper, top=2.5cm}
% \footskip = 60pt

% \setmainfont{TimesNewRomanPSMT}
% \setsansfont{Helvetica-Light}
\setCJKmainfont[ItalicFont=STKaitiSC-Regular,BoldFont=SimSong-Bold]{SimSong-Regular}
\setCJKsansfont[BoldFont=STHeitiSC-Medium]{STHeitiSC-Light}


%\setmainfont{Times New Roman}

\ctexset{today=old}%日期类型设置

% ======================================
% = Color de la Universidad de Sevilla =
% ======================================
\usepackage{tikz}
\definecolor{PKUred}{cmyk}{0,1,1,0.45}

%超链接设置
\usepackage[breaklinks,colorlinks,linkcolor=PKUred,citecolor=PKUred,pagebackref,urlcolor=PKUred]{hyperref}
\usepackage{cleveref}
\newcommand{\crefpairconjunction}{ 和 }
% \newcommand{\crefmiddleconjunction}{ 和 } 
\newcommand{\creflastconjunction}{ 和 }

\newcommand{\hsp}{\hspace{20pt}}
\newcommand{\nhsp}{\hspace{-30pt}}
% \titleformat{\section}{\Large\bfseries}{%\arabic{section}
% \hspace{-22pt}\textcolor{PKUred}{\vrule width 2pt}\hsp}{0pt}{}


% \titleformat{\subsection}
% {\normalfont\large\bfseries}{}{0em}{}

\renewcommand*\footnoterule{%
   \vspace*{-3pt}%
   {\color{PKUred}\hrule width 2in height 0.4pt}%
   \vspace*{2.6pt}%
}


%% Color the bullets of the itemize environment and make the symbol of the third
%% level a diamond instead of an asterisk.
%h\renewcommand*\textbullet{\dag}
\renewcommand*\labelitemi{\color{PKUred}\textbullet}
\renewcommand*\labelitemii{\color{PKUred}--}
\renewcommand*\labelitemiii{\color{PKUred}$\diamond$}
\renewcommand*\labelitemiv{\color{PKUred}\textperiodcentered}



%%% Equation and float numbering
\numberwithin{equation}{section}		% Equationnumbering: section.eq#
\numberwithin{figure}{section}			% Figurenumbering: section.fig#
\numberwithin{table}{section}				% Tablenumbering: section.tab#


%代码设置
\usepackage{listings}
\usepackage{accsupp}
\usepackage{fontspec} % 定制字体
\newfontfamily\menlo{Menlo-Regular}
\usepackage{xcolor} % 定制颜色
\definecolor{mygreen}{rgb}{0,0.6,0}
\definecolor{mygray}{rgb}{0.5,0.5,0.5}
\definecolor{mymauve}{rgb}{0.58,0,0.82}
\lstset{
   numbers=left,
   numberstyle=\footnotesize\menlo,
   basicstyle=\footnotesize\menlo,
   backgroundcolor=\color{white},      % choose the background color
   columns=fullflexible,
   tabsize=4,
   breaklines=true,               % automatic line breaking only at whitespace
   captionpos=b,                  % sets the caption-position to bottom
   commentstyle=\color{mygreen},  % comment style
   escapeinside={\%*}{*)},        % if you want to add LaTeX within your code
   keywordstyle=\color{blue},     % keyword style
   stringstyle=\color{mymauve}\ttfamily,  % string literal style
   frame=lrtb,
   rulesepcolor=\color{red!20!green!20!blue!20},
   % identifierstyle=\color{red},
   % language=c++,
   xleftmargin=4em,xrightmargin=2em, aboveskip=1em,
   % framexleftmargin=2em,
   numbers=left
}

%脚注
\renewcommand\thefootnote{\fnsymbol{footnote}}

%定义常数i、e、积分符号d
\newcommand\mi{\mathrm{i}}
\newcommand\me{\mathrm{e}}

%%% Maketitle metadata
\newcommand{\horrule}[1]{\rule{\linewidth}{#1}} 	% Horizontal rule
\newcommand{\tabincell}[2]{\begin{tabular}{@{}#1@{}}#2\end{tabular}}


\setcounter{secnumdepth}{2}
\usepackage{bm}
\usepackage{autobreak}
\usepackage{amsmath}
\setlength{\parindent}{2em}
\graphicspath{{../code/}}


%pdf 文件设置
\hypersetup{
   pdfauthor={袁磊祺},
   pdftitle={湍流4}
}

\title{
   \vspace{-1in}
   \usefont{OT1}{bch}{b}{n}
   \normalfont \normalsize \textsc{\LARGE Peking University}\\[1cm] % Name of your university/college \\ [25pt]
   \horrule{0.5pt} \\[0.5cm]
   \huge \bfseries{湍流4} \\
   \horrule{2pt} \\[0.5cm]
}
\author{
   \normalfont									\normalsize
   College of Engineering \quad 2001111690  \quad 袁磊祺\\	\normalsize
   \today
}
\date{}

\begin{document}

%%%%%%%%%%%%%%%%%%%%%%%%%%%%%%%%%%%%%%%%%%%%%%
\captionsetup[figure]{name={图},labelsep=period}
\captionsetup[table]{name={表},labelsep=period}
\renewcommand\contentsname{目录}
\renewcommand\listfigurename{插图目录}
\renewcommand\listtablename{表格目录}
\renewcommand\refname{参考文献}
\renewcommand\indexname{索引}
\renewcommand\figurename{图}
\renewcommand\tablename{表}
\renewcommand\abstractname{摘\quad 要}
\renewcommand\partname{部分}
\renewcommand\appendixname{附录}
\def\equationautorefname{式}%
\def\footnoteautorefname{脚注}%
\def\itemautorefname{项}%
\def\figureautorefname{图}%
\def\tableautorefname{表}%
\def\partautorefname{篇}%
\def\appendixautorefname{附录}%
\def\chapterautorefname{章}%
\def\sectionautorefname{节}%
\def\subsectionautorefname{小小节}%
\def\subsubsectionautorefname{subsubsection}%
\def\paragraphautorefname{段落}%
\def\subparagraphautorefname{子段落}%
\def\FancyVerbLineautorefname{行}%
\def\theoremautorefname{定理}%
\crefname{figure}{图}{图}
\crefname{equation}{式}{式}
\crefname{table}{表}{表}
%%%%%%%%%%%%%%%%%%%%%%%%%%%%%%%%%%%%%%%%%%%

\maketitle

2021 年 1 月 3 日前交电子版.

代码等作业内容可在 \texttt{\href{https://github.com/circlelq/Turbulence}{https://github.com/circlelq/Turbulence}} 查看。


\section{1}

自己构造一个在适当的小尺度范围满足 “ $2 / 3$ ” 定律的二阶纵向速度结构函数, 画出结构函数和一维能谱, 考察能谱函数是否满足 “ $5 / 3$ ” 定律。

\textsf{\hspace{-2em}\sf  \textbf{解:}}


\section{2}

考虑均匀各向同性湍流早期的衰减过程, 用 Pao (鲍亦和) 的能量传递模型和 相似性解数值求解无量纲的三维能谱函数, 并在一张图上画出若干初始雷诺数 下的无量纲能谱函数曲线。


\section{3}

阅读 Kolmogorov 1941 年的三篇湍流经典论文(见分享的书 Turbulence: Classical papers on statistical theory) 。 对较少关注的第二篇论文, 在你阅读后用 最简洁的论证给出 “$-10/7$” 衰减规律。并问用类似的方法, 通过适当修改假设, 可否推出湍流动能的指数规律衰减规律 (有人在一定形状的分形格栅尾流中测 量到此规律)?

\section{5}

K62 模型中假设湍流在尺度 $\ell$ 上的粗粒化耗散率 $\varepsilon_{\ell}$ 满足对数正态分布, 并且均
值和方差也满足随尺度变化的一定的对数规律, 由此推出 $p$ 阶速度结构函数的
标度指数规律为 $\zeta(p)=\frac{p}{3}+\frac{\mu p}{18}(3-p)$ 。

证明: 尺度 $\ell$ 上速度增量的绝对值 $U$ 作为随机变量, 其概率密度 $P(\ell, U)$ 也符合对数正态分布, 且满足如下方程:
\begin{equation}
   \ell \frac{\partial P}{\partial \ell}+(\gamma+4 b) U \frac{\partial P}{\partial U}+b U^{2} \frac{\partial^{2} P}{\partial U^{2}}+(\gamma+2 b) P=0,
\end{equation}
其中 $\gamma=(3+\mu) / 9, b=\mu / 18$ 。


\section{6}

试估计用大涡模拟方法计算高雷诺数边界层湍流时的网格规模。参考 Pope 书
习题 13.29 (A) 推出 (13.173) 式。


\textsf{\hspace{-2em}\sf  \textbf{解:}}

In LES-NWR, in order to resolve the near-wall motions, the filter width and grid spacing in the viscous near-wall region must be on the order of $\delta_{\nu}$.\cite{pop} So 
\begin{equation}
   \Delta x=a_{x} \delta_{v}, \quad \Delta z=a_{z} \delta_{v}.
\end{equation}
where $a_x$ and $a_z$ are positive constants.\qed



\section{7}

Prandtl 混合长模型是应用最多的一个湍流模型。但教科书中通常都是针对平
均流动为二维单向的剪切湍流给出给模型。试给出混合长模型的一个三维各向
异性推广。注意: 模型应符合坐标不变的张量性质。


\section{8}

推导周期边界条件下 Navier--Stokes 方程在傅立叶谱空间中的形式。从数学形 式上看, 二维和三维情况下最后有区别吗? 用周期边界条件模拟高雷诺数均匀 各向同性湍流, 在模拟时计算参数的选取应注意什么问题?











% \nocite{*}

% \newpage
\bibliographystyle{plain}
% \clearpage
\phantomsection

\addcontentsline{toc}{section}{参考文献} %向目录中添加条目,以章的名义
\bibliography{homework}

\end{document}
