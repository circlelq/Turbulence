\documentclass[12pt]{ctexart}
\input{setting.tex}
\setcounter{secnumdepth}{2}
\usepackage{bm}
\usepackage{autobreak}
\usepackage{amsmath}
% \setlength{\parindent}{2em}
% \graphicspath{{fig/}}


%pdf 文件设置
\hypersetup{
   pdfauthor={袁磊祺},
   pdftitle={Notes}
}

\title{
   \vspace{-1in} 	
   \usefont{OT1}{bch}{b}{n}
   \normalfont \normalsize \textsc{\LARGE Peking University}\\[1cm] % Name of your university/college \\ [25pt]
   \horrule{0.5pt} \\[0.5cm]
   \huge \bfseries{Notes} \\
   \horrule{2pt} \\[0.5cm]
}
\author{
   \normalfont 								\normalsize
   College of Engineering \quad 2001111690  \quad 袁磊祺\\	\normalsize
   \today
}
\date{}

\begin{document}

% \input{setc.tex}

\maketitle

\section{卡门--豪沃思方程}


N--S 
\begin{equation}
   \frac{\partial R_{ij}}{\partial t} - \frac{\partial }{\partial r_m} \left( S_{imj} + S_{jmi} \right) = 2 \nu \nabla^2 R_{ij}
\end{equation}
K--H
\begin{equation}
   \frac{\partial \left( \left<	u^2 \right> f \right) }{\partial t}  - \left< u^2 \right> ^{\frac{3}{2}} \left( k' + \frac{4k}{r} \right) = 2 \nu \left< u^2 \right> \left( f'' + \frac{4f'}{r} \right) 
\end{equation}
自模拟假设(自保持,自相似)$f(r,t),\, k(r,t),\, \left<	u^2 \right>(t)$ 引入(微尺度)$\lambda(t)$ 

设 $f(r,t) = F\left( \frac{r}{\lambda(t)} \right) ,\quad k(r,t) = K\left( \frac{r}{\lambda (t)} \right) $
\begin{equation}
   \frac{\partial \left( r^4 \left<	u^2 \right> f \right) }{\partial t}  - \left< u^2 \right> ^{\frac{3}{2}} \left(r^4 k\right)' = 2 \nu \left< u^2 \right> \left(r^4f'\right)'
\end{equation}
\begin{equation}
   \frac{\partial \int_{0}^{+\infty}\left( r^4 \left<	u^2 \right> f \right) \dif r }{\partial t}  - \left< u^2 \right> ^{\frac{3}{2}} \left(r^4 k\right)  \big|_{0}^{+\infty} = 2 \nu \left< u^2 \right> \left(r^4f'\right)\big|_{0}^{+\infty}
\end{equation}
Loitsansky 积分不变量
\begin{equation}
   \left< u^2 \right> \int_{0}^{\infty} r^4 f \dif r = \Lambda_0
\end{equation}
与时间无关的假设, $r^4 f' \to 0$.
Taylor 展开结果:
\begin{equation}
   f = 1 - \frac{1}{2} \left( \frac{r}{\lambda} \right)^2 + \frac{f''''(0)}{24}r^4 + \cdots
\end{equation}
\begin{equation}
   k = \frac{k'''(0)}{6}r^3 + \cdots
\end{equation}
\begin{equation}
   \frac{\partial \left< u^2 \right>f}{\partial t}  = \frac{\dif \left< u^2 \right>}{\dif t} - \frac{1}{2} \left( \frac{r}{\lambda} \right)^2 \frac{\dif \left< u^2 \right>}{\dif t} + \frac{r^2}{\lambda^3} \frac{\dif \lambda}{\dif t} + \cdots
\end{equation}
\begin{equation}
   k' + \frac{4k}{r} =  \frac{k'''(0)}{2} r^2 + \frac{2}{3}k'''(0) r^2 + \cdots = \frac{7}{6} k'''(0) r^2 + \cdots
\end{equation}
\begin{equation}
   f'' + \frac{4f'}{r} = - \frac{1}{\lambda^2} - \frac{4}{\lambda^2} + \cdots = -\frac{5}{\lambda^2} + \cdots
\end{equation}
\begin{equation}
   \frac{\dif \left< u^2 \right>}{\dif t} = -10 \nu \frac{\left< u^2 \right>}{\lambda^2}
\end{equation}
\begin{equation}
   \frac{\dif }{\dif t} \left< \omega^2 \right> = \frac{7}{3\sqrt{15}} \left< \omega^2 \right>^{\frac{3}{2}} \left( S - \frac{2G}{R_{\lambda}} \right) 
\end{equation}
\begin{equation}
   S \equiv - \lambda^3 k'''(0),\quad G \equiv \lambda^4 f''''(0),\quad R_{\lambda} \equiv \frac{\left< u^2 \right>^{\frac{1}{2}}\lambda}{\nu}
\end{equation}

\section{HIT衰变后期规律}
$k$ 忽略(惯性项近似为$0$ )
\begin{equation}
   \frac{\partial \left< u^2 \right>f}{\partial t}  = 2\nu \left<u^2 \right> \left( f'' + \frac{4f'}{r} \right) 
\end{equation}
设 $f = F\left( \frac{r}{\lambda(t)} \right) $,则$\xi \equiv \frac{r}{\lambda(t)}$
\begin{equation}
   \frac{\dif \left<u^2 \right>}{\dif t} F + \left< u^2 \right> F' \left( - \frac{\xi}{\lambda} \frac{\dif  \lambda}{\dif t} \right) = 2 \nu \left< u^2 \right>  \left( \frac{F''}{\lambda^2} + \frac{4F'}{r\lambda} \right) 
\end{equation}
\begin{equation}
   -10 \nu \frac{\left< u^2 \right>}{\lambda^2} \left( -\xi \lambda \frac{\dif \lambda}{\dif t} \right) F'
\end{equation}
\begin{equation}
   -10 \nu F = 2 \nu \left( F'' + \frac{4F'}{\xi} \right)  + F' \xi \lambda \frac{\dif \lambda}{\dif t}
\end{equation}
\begin{equation}
   F'' + \frac{4F'}{\xi} + F' \xi \frac{\lambda}{2\nu} \frac{\dif \lambda}{\dif t} + 5F = 0
\end{equation}
\begin{equation}
   \alpha = \frac{\lambda}{2\nu} \frac{\dif \lambda}{\dif t} = \frac{1}{4\nu} \frac{\dif  \lambda^2}{\dif t} \implies \text{常数}
\end{equation}
\begin{equation}
   \lambda^2 = 4\nu (t-t_0) \alpha
\end{equation}
\begin{equation}
   \frac{\dif \left<u^2 \right>}{\dif t} = -10 \nu \frac{\left< u^2 \right>}{4\nu\alpha(t-t_0)}
\end{equation}
\begin{equation}
   \left< u^2 \right> = A (t-t_0)^{-\frac{5}{2\alpha}}
\end{equation}
\begin{equation}
   \left<u^2 \right> \int_{0}^{+\infty} r^4 f \dif r = \left< u^2 \right> \lambda^5 \int_0^{+\infty} \xi^5 F(\xi) \dif \xi = \Lambda_0 \implies \alpha = 1.
\end{equation}
\begin{equation}
   \left< u^2 \right> \sim (t-t_0)^{\frac{5}{2}}
\end{equation}
\begin{equation}
   F'' + \left( \xi + \frac{4}{\xi} \right) F' + 5F = 0
\end{equation}
设$x = \xi^2,\quad F(\xi) = y(x)$
\begin{equation}
   F' = y' 2\xi
\end{equation}
\begin{equation}
   F'' = 2y' + y'' 4x
\end{equation}
\begin{equation}
   4xy'' + \left( \xi + \frac{4}{\xi} \right) 2\xi y' + 5y = 0
\end{equation}
\begin{equation}
   4xy'' + (2x + 10)y' + 5y = 0
\end{equation}
设$y=\me^{` x`}$
\begin{equation}
   y' = \beta y,\quad y'' = \beta^2 y
\end{equation}
\begin{equation}
   4x\beta^2 y + (2x+8) \beta y + 5y =0
\end{equation}
\begin{equation}
   4x\beta^2 y + 2x\beta + 10\beta + 5 = 0
\end{equation}
\begin{equation}
   4\beta^2 + 2\beta = 0 \implies \beta = - \frac{1}{2}
\end{equation}
\begin{equation}
   F(\xi) = C \me^{-\frac{\xi^2}{2}}
\end{equation}
\begin{equation}
   f(r,t) = C \me^{-\frac{1}{2} \frac{r^2}{4\nu(t-t_0)}} = C \me^{-\frac{r^2}{8\nu(t-t_0)}} \implies f = \me^{-\frac{r^2}{8\nu(t-t_0)}}
\end{equation}
谱空间
\begin{equation}
   \frac{\partial }{\partial t} \Phi_{ij} - \mi m_m \left( \Gamma_{imj} + \Gamma_{jmi} \right) = -2 \nu k^2 \Phi_{i} 
\end{equation}
缩并 $i,$
\begin{equation}
   \Phi_{ij} = \frac{E}{4\pi k^4}(k^2\delta_{ij}-k_i k_j)
\end{equation}
\begin{equation}
   \Gamma_{ijl} = \mi \Gamma \left( k_i k_j k_l - \frac{k^2}{2} (k_i\delta_{jl} + k_j \delta_{il}) \right) 
\end{equation}
\begin{equation}
   \Gamma_{imi} = \mi \Gamma \left( k^2k_m - \frac{k^2}{2} (k_m + 3k_m) \right) = - \mi k^2 \Gamma k_m
\end{equation}
\begin{equation}
   \frac{1}{2\pi k^2} \frac{\partial E}{\partial t}  - 2k^4 \Gamma = -2 \nu k^2 \frac{E}{2\pi k^2}
\end{equation}
\begin{equation}
   \left( \frac{\partial }{\partial t} + 2\nu k^2 \right) E = 4\pi k^6 \Gamma \equiv T(k,t) 
\end{equation}
\begin{equation}
   \frac{\partial }{\partial t} \int_0^k E(k,t) \dif k - \int_0^k T(k,t)\dif k = -2 \nu \int_0^k k^2 E \dif k \equiv \Pi(k,t) = \int_k^{\infty} T \dif k
\end{equation}
$T$ 忽略
\begin{equation}
   \frac{\partial E}{\partial t} = -2 \nu k^2 E \implies E(k,t) = E_0(k) \me^{-2\nu k^2t}
\end{equation}
设$E = V^2 l F(kl)$ 用 Loistansky 不变量可得$ E_0(k) = ck^4$

Kolmogorov 1941 理论

\begin{enumerate}
   \item 展示湍流(一般湍流)在Re极大时,在局部为均匀各向同性(远离边界,奇点),在增量意义下。$n$ 点联合p.d.f(增量)只依赖于$n$ 点构型形状、大小与位置、时刻及方位无关。由此引入结构函数的概念。
   \item 在Re极大时,小尺度范围为普适平衡。$ \frac{\partial }{\partial t} \left<\cdot \right> \approx 0$,统计特性$\left<\cdot \right>$ 只依赖于$\left<\varepsilon\right>,\nu$
   \item 在Re极大时,小尺度范围的低波数段,统计量只依赖于$\left<\epsilon \right>$ (第二相似性假设)
\end{enumerate}
\begin{equation}
   S_2(r) = B(\epsilon, \nu, r)
\end{equation}
\begin{equation}
   \eta \equiv \left( \frac{\nu^3}{\epsilon} \right) ^{\frac{1}{4}},\quad v \equiv (\epsilon \eta)^{\frac{1}{3}}
\end{equation}
\begin{equation}
   S_2(r) = v^2 F \left( \frac{r}{\eta} \right),\quad S_3(r) = v^3 G(\frac{r}{\eta}),\quad \frac{\eta v}{\nu} = 1
\end{equation}
\begin{equation}
   S_2 = B(r,\epsilon),\quad v \sim (\epsilon r)^{\frac{1}{3}},\quad S_2 = C_2 (\epsilon r)^{\frac{2}{3}}
\end{equation}











% \nocite{*}

% \input{bib.tex}

\end{document}
