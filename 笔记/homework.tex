\documentclass[12pt]{ctexart}
\usepackage{natbib}
\usepackage{url}
\usepackage{stmaryrd}
\usepackage{mathrsfs}
\usepackage{amsmath}
\usepackage{graphicx}
\usepackage{parskip}
\usepackage{fancyhdr}
% \usepackage{underscore} % 下划线
\usepackage{commath}%定义d
\usepackage{geometry}
\usepackage{bm}
\usepackage{siunitx}
\usepackage{float}
% \usepackage[skip=-5pt]{subcaption}
% \usepackage{subfig}
\usepackage{subfigure}  %插入多图时用子图显示的宏包
\usepackage{titlesec}
\usepackage{caption}
\usepackage{paralist}
\usepackage{multirow}
\usepackage{booktabs} % To thicken table lines
\usepackage{diagbox}
\usepackage{authblk}
\usepackage{indentfirst}
\usepackage{amsthm}
\usepackage{fontspec}
\usepackage{color}
%\usepackage{txfonts} %设置字体为times new roman
\usepackage{lettrine}
\usepackage{nameref}
%\usepackage[nottoc]{tocbibind}
\usepackage{amssymb}%font
\usepackage{lipsum}%make test words
\usepackage{picinpar}%words around the picture
\usepackage[all]{xy}%draw arrow
\usepackage{asymptote}%draw picture
\usepackage[perpage]{footmisc}%脚注每页清零
\usepackage{esint}
\renewcommand{\proofname}{\indent \sf \bfseries{证明}}
\pagestyle{fancy}
\fancyhf{}
\renewcommand{\headrulewidth}{0pt}
\fancyfoot[C]{\thepage}

\catcode`\。=\active
\catcode`\,=\active
\catcode`\;=\active
\catcode`\:=\active
\newcommand{。}{.}
\newcommand{,}{,}
\newcommand{;}{;}
\newcommand{:}{:}

\geometry{bottom=3cm,left=3cm,right=3cm,a4paper, top=2.5cm}
% \footskip = 60pt

% \setmainfont{TimesNewRomanPSMT}
% \setsansfont{Helvetica-Light}
\setCJKmainfont[ItalicFont=STKaitiSC-Regular,BoldFont=SimSong-Bold]{SimSong-Regular}
\setCJKsansfont[BoldFont=STHeitiSC-Medium]{STHeitiSC-Light}


%\setmainfont{Times New Roman}

\ctexset{today=old}%日期类型设置

% ======================================
% = Color de la Universidad de Sevilla =
% ======================================
\usepackage{tikz}
\definecolor{PKUred}{cmyk}{0,1,1,0.45}

%超链接设置
\usepackage[breaklinks,colorlinks,linkcolor=PKUred,citecolor=PKUred,pagebackref,urlcolor=PKUred]{hyperref}
\usepackage{cleveref}
\newcommand{\crefpairconjunction}{ 和 }
% \newcommand{\crefmiddleconjunction}{ 和 } 
\newcommand{\creflastconjunction}{ 和 }

\newcommand{\hsp}{\hspace{20pt}}
\newcommand{\nhsp}{\hspace{-30pt}}
% \titleformat{\section}{\Large\bfseries}{%\arabic{section}
% \hspace{-22pt}\textcolor{PKUred}{\vrule width 2pt}\hsp}{0pt}{}


% \titleformat{\subsection}
% {\normalfont\large\bfseries}{}{0em}{}

\renewcommand*\footnoterule{%
   \vspace*{-3pt}%
   {\color{PKUred}\hrule width 2in height 0.4pt}%
   \vspace*{2.6pt}%
}


%% Color the bullets of the itemize environment and make the symbol of the third
%% level a diamond instead of an asterisk.
%h\renewcommand*\textbullet{\dag}
\renewcommand*\labelitemi{\color{PKUred}\textbullet}
\renewcommand*\labelitemii{\color{PKUred}--}
\renewcommand*\labelitemiii{\color{PKUred}$\diamond$}
\renewcommand*\labelitemiv{\color{PKUred}\textperiodcentered}



%%% Equation and float numbering
\numberwithin{equation}{section}		% Equationnumbering: section.eq#
\numberwithin{figure}{section}			% Figurenumbering: section.fig#
\numberwithin{table}{section}				% Tablenumbering: section.tab#


%代码设置
\usepackage{listings}
\usepackage{accsupp}
\usepackage{fontspec} % 定制字体
\newfontfamily\menlo{Menlo-Regular}
\usepackage{xcolor} % 定制颜色
\definecolor{mygreen}{rgb}{0,0.6,0}
\definecolor{mygray}{rgb}{0.5,0.5,0.5}
\definecolor{mymauve}{rgb}{0.58,0,0.82}
\lstset{
   numbers=left,
   numberstyle=\footnotesize\menlo,
   basicstyle=\footnotesize\menlo,
   backgroundcolor=\color{white},      % choose the background color
   columns=fullflexible,
   tabsize=4,
   breaklines=true,               % automatic line breaking only at whitespace
   captionpos=b,                  % sets the caption-position to bottom
   commentstyle=\color{mygreen},  % comment style
   escapeinside={\%*}{*)},        % if you want to add LaTeX within your code
   keywordstyle=\color{blue},     % keyword style
   stringstyle=\color{mymauve}\ttfamily,  % string literal style
   frame=lrtb,
   rulesepcolor=\color{red!20!green!20!blue!20},
   % identifierstyle=\color{red},
   % language=c++,
   xleftmargin=4em,xrightmargin=2em, aboveskip=1em,
   % framexleftmargin=2em,
   numbers=left
}

%脚注
\renewcommand\thefootnote{\fnsymbol{footnote}}

%定义常数i、e、积分符号d
\newcommand\mi{\mathrm{i}}
\newcommand\me{\mathrm{e}}

%%% Maketitle metadata
\newcommand{\horrule}[1]{\rule{\linewidth}{#1}} 	% Horizontal rule
\newcommand{\tabincell}[2]{\begin{tabular}{@{}#1@{}}#2\end{tabular}}


\setcounter{secnumdepth}{2}
\usepackage{bm}
\usepackage{autobreak}
\usepackage{amsmath}
% \setlength{\parindent}{2em}
% \graphicspath{{fig/}}


%pdf 文件设置
\hypersetup{
   pdfauthor={袁磊祺},
   pdftitle={Notes}
}

\title{
   \vspace{-1in} 	
   \usefont{OT1}{bch}{b}{n}
   \normalfont \normalsize \textsc{\LARGE Peking University}\\[1cm] % Name of your university/college \\ [25pt]
   \horrule{0.5pt} \\[0.5cm]
   \huge \bfseries{Notes} \\
   \horrule{2pt} \\[0.5cm]
}
\author{
   \normalfont 								\normalsize
   College of Engineering \quad 2001111690  \quad 袁磊祺\\	\normalsize
   \today
}
\date{}

\begin{document}

% %%%%%%%%%%%%%%%%%%%%%%%%%%%%%%%%%%%%%%%%%%%%%%
\captionsetup[figure]{name={图},labelsep=period}
\captionsetup[table]{name={表},labelsep=period}
\renewcommand\contentsname{目录}
\renewcommand\listfigurename{插图目录}
\renewcommand\listtablename{表格目录}
\renewcommand\refname{参考文献}
\renewcommand\indexname{索引}
\renewcommand\figurename{图}
\renewcommand\tablename{表}
\renewcommand\abstractname{摘\quad 要}
\renewcommand\partname{部分}
\renewcommand\appendixname{附录}
\def\equationautorefname{式}%
\def\footnoteautorefname{脚注}%
\def\itemautorefname{项}%
\def\figureautorefname{图}%
\def\tableautorefname{表}%
\def\partautorefname{篇}%
\def\appendixautorefname{附录}%
\def\chapterautorefname{章}%
\def\sectionautorefname{节}%
\def\subsectionautorefname{小小节}%
\def\subsubsectionautorefname{subsubsection}%
\def\paragraphautorefname{段落}%
\def\subparagraphautorefname{子段落}%
\def\FancyVerbLineautorefname{行}%
\def\theoremautorefname{定理}%
\crefname{figure}{图}{图}
\crefname{equation}{式}{式}
\crefname{table}{表}{表}
%%%%%%%%%%%%%%%%%%%%%%%%%%%%%%%%%%%%%%%%%%%

\maketitle

\section{卡门--豪沃思方程}


N--S 
\begin{equation}
   \frac{\partial R_{ij}}{\partial t} - \frac{\partial }{\partial r_m} \left( S_{imj} + S_{jmi} \right) = 2 \nu \nabla^2 R_{ij}
\end{equation}
K--H
\begin{equation}
   \frac{\partial \left( \left<	u^2 \right> f \right) }{\partial t}  - \left< u^2 \right> ^{\frac{3}{2}} \left( k' + \frac{4k}{r} \right) = 2 \nu \left< u^2 \right> \left( f'' + \frac{4f'}{r} \right) 
\end{equation}
自模拟假设(自保持,自相似)$f(r,t),\, k(r,t),\, \left<	u^2 \right>(t)$ 引入(微尺度)$\lambda(t)$ 

设 $f(r,t) = F\left( \frac{r}{\lambda(t)} \right) ,\quad k(r,t) = K\left( \frac{r}{\lambda (t)} \right) $
\begin{equation}
   \frac{\partial \left( r^4 \left<	u^2 \right> f \right) }{\partial t}  - \left< u^2 \right> ^{\frac{3}{2}} \left(r^4 k\right)' = 2 \nu \left< u^2 \right> \left(r^4f'\right)'
\end{equation}
\begin{equation}
   \frac{\partial \int_{0}^{+\infty}\left( r^4 \left<	u^2 \right> f \right) \dif r }{\partial t}  - \left< u^2 \right> ^{\frac{3}{2}} \left(r^4 k\right)  \big|_{0}^{+\infty} = 2 \nu \left< u^2 \right> \left(r^4f'\right)\big|_{0}^{+\infty}
\end{equation}
Loitsansky 积分不变量
\begin{equation}
   \left< u^2 \right> \int_{0}^{\infty} r^4 f \dif r = \Lambda_0
\end{equation}
与时间无关的假设, $r^4 f' \to 0$.
Taylor 展开结果:
\begin{equation}
   f = 1 - \frac{1}{2} \left( \frac{r}{\lambda} \right)^2 + \frac{f''''(0)}{24}r^4 + \cdots
\end{equation}
\begin{equation}
   k = \frac{k'''(0)}{6}r^3 + \cdots
\end{equation}
\begin{equation}
   \frac{\partial \left< u^2 \right>f}{\partial t}  = \frac{\dif \left< u^2 \right>}{\dif t} - \frac{1}{2} \left( \frac{r}{\lambda} \right)^2 \frac{\dif \left< u^2 \right>}{\dif t} + \frac{r^2}{\lambda^3} \frac{\dif \lambda}{\dif t} + \cdots
\end{equation}
\begin{equation}
   k' + \frac{4k}{r} =  \frac{k'''(0)}{2} r^2 + \frac{2}{3}k'''(0) r^2 + \cdots = \frac{7}{6} k'''(0) r^2 + \cdots
\end{equation}
\begin{equation}
   f'' + \frac{4f'}{r} = - \frac{1}{\lambda^2} - \frac{4}{\lambda^2} + \cdots = -\frac{5}{\lambda^2} + \cdots
\end{equation}
\begin{equation}
   \frac{\dif \left< u^2 \right>}{\dif t} = -10 \nu \frac{\left< u^2 \right>}{\lambda^2}
\end{equation}
\begin{equation}
   \frac{\dif }{\dif t} \left< \omega^2 \right> = \frac{7}{3\sqrt{15}} \left< \omega^2 \right>^{\frac{3}{2}} \left( S - \frac{2G}{R_{\lambda}} \right) 
\end{equation}
\begin{equation}
   S \equiv - \lambda^3 k'''(0),\quad G \equiv \lambda^4 f''''(0),\quad R_{\lambda} \equiv \frac{\left< u^2 \right>^{\frac{1}{2}}\lambda}{\nu}
\end{equation}

\section{HIT衰变后期规律}
$k$ 忽略(惯性项近似为$0$ )
\begin{equation}
   \frac{\partial \left< u^2 \right>f}{\partial t}  = 2\nu \left<u^2 \right> \left( f'' + \frac{4f'}{r} \right) 
\end{equation}
设 $f = F\left( \frac{r}{\lambda(t)} \right) $,则$\xi \equiv \frac{r}{\lambda(t)}$
\begin{equation}
   \frac{\dif \left<u^2 \right>}{\dif t} F + \left< u^2 \right> F' \left( - \frac{\xi}{\lambda} \frac{\dif  \lambda}{\dif t} \right) = 2 \nu \left< u^2 \right>  \left( \frac{F''}{\lambda^2} + \frac{4F'}{r\lambda} \right) 
\end{equation}
\begin{equation}
   -10 \nu \frac{\left< u^2 \right>}{\lambda^2} \left( -\xi \lambda \frac{\dif \lambda}{\dif t} \right) F'
\end{equation}
\begin{equation}
   -10 \nu F = 2 \nu \left( F'' + \frac{4F'}{\xi} \right)  + F' \xi \lambda \frac{\dif \lambda}{\dif t}
\end{equation}
\begin{equation}
   F'' + \frac{4F'}{\xi} + F' \xi \frac{\lambda}{2\nu} \frac{\dif \lambda}{\dif t} + 5F = 0
\end{equation}
\begin{equation}
   \alpha = \frac{\lambda}{2\nu} \frac{\dif \lambda}{\dif t} = \frac{1}{4\nu} \frac{\dif  \lambda^2}{\dif t} \implies \text{常数}
\end{equation}
\begin{equation}
   \lambda^2 = 4\nu (t-t_0) \alpha
\end{equation}
\begin{equation}
   \frac{\dif \left<u^2 \right>}{\dif t} = -10 \nu \frac{\left< u^2 \right>}{4\nu\alpha(t-t_0)}
\end{equation}
\begin{equation}
   \left< u^2 \right> = A (t-t_0)^{-\frac{5}{2\alpha}}
\end{equation}
\begin{equation}
   \left<u^2 \right> \int_{0}^{+\infty} r^4 f \dif r = \left< u^2 \right> \lambda^5 \int_0^{+\infty} \xi^5 F(\xi) \dif \xi = \Lambda_0 \implies \alpha = 1.
\end{equation}
\begin{equation}
   \left< u^2 \right> \sim (t-t_0)^{\frac{5}{2}}
\end{equation}
\begin{equation}
   F'' + \left( \xi + \frac{4}{\xi} \right) F' + 5F = 0
\end{equation}
设$x = \xi^2,\quad F(\xi) = y(x)$
\begin{equation}
   F' = y' 2\xi
\end{equation}
\begin{equation}
   F'' = 2y' + y'' 4x
\end{equation}
\begin{equation}
   4xy'' + \left( \xi + \frac{4}{\xi} \right) 2\xi y' + 5y = 0
\end{equation}
\begin{equation}
   4xy'' + (2x + 10)y' + 5y = 0
\end{equation}
设$y=\me^{` x`}$
\begin{equation}
   y' = \beta y,\quad y'' = \beta^2 y
\end{equation}
\begin{equation}
   4x\beta^2 y + (2x+8) \beta y + 5y =0
\end{equation}
\begin{equation}
   4x\beta^2 y + 2x\beta + 10\beta + 5 = 0
\end{equation}
\begin{equation}
   4\beta^2 + 2\beta = 0 \implies \beta = - \frac{1}{2}
\end{equation}
\begin{equation}
   F(\xi) = C \me^{-\frac{\xi^2}{2}}
\end{equation}
\begin{equation}
   f(r,t) = C \me^{-\frac{1}{2} \frac{r^2}{4\nu(t-t_0)}} = C \me^{-\frac{r^2}{8\nu(t-t_0)}} \implies f = \me^{-\frac{r^2}{8\nu(t-t_0)}}
\end{equation}
谱空间
\begin{equation}
   \frac{\partial }{\partial t} \Phi_{ij} - \mi m_m \left( \Gamma_{imj} + \Gamma_{jmi} \right) = -2 \nu k^2 \Phi_{i} 
\end{equation}
缩并 $i,$
\begin{equation}
   \Phi_{ij} = \frac{E}{4\pi k^4}(k^2\delta_{ij}-k_i k_j)
\end{equation}
\begin{equation}
   \Gamma_{ijl} = \mi \Gamma \left( k_i k_j k_l - \frac{k^2}{2} (k_i\delta_{jl} + k_j \delta_{il}) \right) 
\end{equation}
\begin{equation}
   \Gamma_{imi} = \mi \Gamma \left( k^2k_m - \frac{k^2}{2} (k_m + 3k_m) \right) = - \mi k^2 \Gamma k_m
\end{equation}
\begin{equation}
   \frac{1}{2\pi k^2} \frac{\partial E}{\partial t}  - 2k^4 \Gamma = -2 \nu k^2 \frac{E}{2\pi k^2}
\end{equation}
\begin{equation}
   \left( \frac{\partial }{\partial t} + 2\nu k^2 \right) E = 4\pi k^6 \Gamma \equiv T(k,t) 
\end{equation}
\begin{equation}
   \frac{\partial }{\partial t} \int_0^k E(k,t) \dif k - \int_0^k T(k,t)\dif k = -2 \nu \int_0^k k^2 E \dif k \equiv \Pi(k,t) = \int_k^{\infty} T \dif k
\end{equation}
$T$ 忽略
\begin{equation}
   \frac{\partial E}{\partial t} = -2 \nu k^2 E \implies E(k,t) = E_0(k) \me^{-2\nu k^2t}
\end{equation}
设$E = V^2 l F(kl)$ 用 Loistansky 不变量可得$ E_0(k) = ck^4$

\subsection{Kolmogorov 1941 理论}

\begin{enumerate}
   \item 展示湍流(一般湍流)在Re极大时,在局部为均匀各向同性(远离边界,奇点),在增量意义下。$n$ 点联合p.d.f(增量)只依赖于$n$ 点构型形状、大小与位置、时刻及方位无关。由此引入结构函数的概念。
   \item 在Re极大时,小尺度范围为普适平衡。$ \frac{\partial }{\partial t} \left<\cdot \right> \approx 0$,统计特性$\left<\cdot \right>$ 只依赖于$\left<\varepsilon\right>,\nu$
   \item 在Re极大时,小尺度范围的低波数段,统计量只依赖于$\left<\varepsilon \right>$ (第二相似性假设)
\end{enumerate}
\begin{equation}
   S_2(r) = B(\varepsilon, \nu, r)
\end{equation}
\begin{equation}
   \eta \equiv \left( \frac{\nu^3}{\varepsilon} \right) ^{\frac{1}{4}},\quad v \equiv (\varepsilon \eta)^{\frac{1}{3}}
\end{equation}
\begin{equation}
   S_2(r) = v^2 F \left( \frac{r}{\eta} \right),\quad S_3(r) = v^3 G(\frac{r}{\eta}),\quad \frac{\eta v}{\nu} = 1
\end{equation}
\begin{equation}
   S_2 = B(r,\varepsilon),\quad v \sim (\varepsilon r)^{\frac{1}{3}},\quad S_2 = C_2 (\varepsilon r)^{\frac{2}{3}}
\end{equation}

衰变后期 $\lambda^2 = 4 \nu (t-t_0)$

纵串(L. F. Richardson 1922) cascade 大涡发展成小涡,非线性作用(惯性)
\begin{equation}
   \begin{aligned}
	  \left< \varepsilon \right>  & = \nu \left<  \abs{\abs{ \nabla \bm{u} }}^2 \right> \\
								  &= \nu \left< \omega^2 \right>  \\
								  &= 15\nu \frac{\left< u^2 \right> }{\lambda^2} \\
   \end{aligned}
\end{equation}
\begin{equation}
   f(r) = 1 - \frac{1}{2} \left( \frac{r}{\lambda} \right)^2 + o(r^2)
\end{equation}

\begin{enumerate}
   \item 局部各向同性
   \item 第一相似性 $\varepsilon, \nu$ 平衡范围
   \item 第二相似性  $\varepsilon$
\end{enumerate}
纵向
\begin{equation}
   S_2  \equiv \left< \left( ( \bm{u} (\bm{x} + \bm{r} ) - \bm{u} (\bm{x} )) \frac{\bm{r} }{r} \right)^2  \right>  = \left< \left( u_2 - u_1 \right)^2 \right> = \left( \varepsilon \eta \right)^{\frac{2}{3}} B_2 (\frac{r}{\eta})
\end{equation}
\begin{equation}
   S_3 \equiv \left< \left( u_2 - u_1 \right)^3 \right> = (\varepsilon \eta)B_3\left( \frac{r}{\eta} \right) 
\end{equation}
\begin{equation}
   S_2 (r) = C_2 (\varepsilon r)^{\frac{2}{3}} \quad \text{实验上基本成立}
\end{equation}
\begin{equation}
   S_3 (r) = C_3 (\varepsilon r)
\end{equation}
$C_2,C_3$ 是普适常数.

Obukhov (1941) 在平衡范围
\begin{equation}
   E(k) = \varepsilon^{\frac{1}{4}} \nu^{\frac{5}{4}} F(k\eta)
\end{equation}
惯性范围 $E(k) = C_k \varepsilon^{\frac{2}{3}} k^{-\frac{5}{3}}$, $-\frac{5}{3}$ 定律$E(k) \sim k^{-\frac{5}{3}}$

K41推广:$p$ 阶矩 $( p \in \mathbb{N} ) $
\begin{equation}
   S_p (r) \sim (\varepsilon r)^{\frac{p}{3}} C_p \implies p_r (\delta u) = ?
\end{equation}
\begin{equation}
   S_p(r) = \left( \frac{r}{L} \right) ^{\frac{p}{3}} S_p(L) \implies p_r (\delta u) = p_L\left(\delta u\left( \frac{r}{L} \right)^{-\frac{1}{3}}\right) \left(\frac{r}{L} \right)^{-\frac{1}{3}}
\end{equation}

K-H: 
\begin{equation}
   \frac{\partial \left( \left< u^2 \right> f \right) }{\partial t}  - \left< u^2 \right>^{\frac{3}{2}} \left( k' + \frac{4k}{r} \right) = 2\nu \left< u^2 \right> \left< f'' + \frac{4f'}{r} \right> 
\end{equation}
$f,k \to (S_2,S_3)$, $S_3 =  \left< u^2 \right>^{\frac{3}{2}}k$
\begin{equation}
   2\left< u^2 \right> -S_2 = 2\left< u^2 \right>  f
\end{equation}
\begin{equation}
   2 \frac{\partial \left< u^2 \right> }{\partial t} - \frac{\partial S_2}{\partial t} - \frac{1}{3} \left( S_3' + \frac{4S_3}{r} \right) = -2 \nu \left( S_2'' + \frac{4S_2'}{r} \right) 
\end{equation}
\begin{equation}
   -\frac{4}{3} \varepsilon - \frac{1}{3} \left( S_3' + \frac{4S_3}{r} \right) = -2 \nu \left( S_2'' + \frac{4S_2'}{r} \right) 
\end{equation}
乘$r^4$
\begin{equation}
   - \frac{r}{3} \varepsilon  r^4 - \frac{1}{3} \left( r^4 S_3 \right) ' = -2 \nu \left( S_2' r^4 \right) '
\end{equation}
积分$r$
\begin{equation}
   - \frac{4}{15} \varepsilon r^5 - \frac{1}{3} r^4 S_3 = -2 \nu S_2' r^4
\end{equation}
\begin{equation}
   S_3 = - \frac{4}{5} \varepsilon r + 6\nu r \frac{\dif S_2}{\dif r}
\end{equation}
当$\nu \to 0$时
\begin{equation}
   S_3(r) = - \frac{4}{5} \varepsilon r
\end{equation}
设一个$S_2$ 的函数,计算$E(k)$
\begin{equation}
   S_2 = C_2 \left( \varepsilon r \right)^{\frac{2}{3}},\quad E(k) = C_k \varepsilon^{\frac{2}{3}} k^{-\frac{5}{3}}
\end{equation}
都是$-\frac{5}{3}$ 的关系
\begin{equation}
   \phi_1(k) = C_1 \varepsilon^{\frac{2}{3}} k^{-\frac{5}{3}}
\end{equation}
\begin{equation}
   C_k = C_1 \left( -\frac{5}{3} \right) \left( - \frac{5}{3} - 2 \right) 
\end{equation}

一维能谱 $\phi_1(k)$
\begin{equation}
   \begin{aligned}
	  \left< u^2 \right> f(r) & = \int_{-\infty}^{+\infty} \phi_1 \me^{\mi kr} \dif k \\
							  &= 2 \int_{0}^{+\infty} \phi_1 \cos k r \dif k\\
   \end{aligned}
\end{equation}
\begin{equation}
   S_2 = 2 \left< u^2 \right>  - 4 \int_{0}^{\infty} \phi_1 \cos kr \dif k 
\end{equation}
\begin{equation}
   \begin{aligned}
	  S_2 & = 4 \int_{0}^{\infty} \phi_1(1- \cos kr) \dif k\\
		  &= 4 \int_{0}^{\infty} C_1 \varepsilon^{\frac{2}{3} k^{-\frac{5}{3}}} (1 - \cos kr ) \dif k \\
		  &= 4C_1 \varepsilon^{\frac{2}{3}} \int_{0}^{\infty} k^{-\frac{5}{3}} (1-\cos kr) \dif k 
   \end{aligned}
\end{equation}
用复变函数积分得
\begin{equation}
   \int_{0}^{\infty} k^{-\frac{5}{3}} \cos kr \dif k  = \frac{\Gamma \left( \frac{1}{3} \right) }{2\sqrt[3]{r} }
\end{equation}
\begin{equation}
   C_2 = 3 \Gamma\left(\frac{1}{3}\right) C_1
\end{equation}

能谱方程
\begin{equation}
   \frac{\partial E}{\partial t} = T - 2 \nu k^2E
\end{equation}
$T(k,t)$ 输运,普适平衡
\begin{equation}
   \begin{aligned}
	  \frac{\partial }{\partial t}  \int_{0}^{k} E \dif k & = \int_{0}^{k} T \dif k - 2 \nu \int_{0}^{k}  k^2 E \dif k \\
														  &= - \int_{k}^{\infty} T \dif k - \left( \varepsilon - 2 \nu \int_{k}^{\infty} k^2 E \dif k  \right)   \\
   \end{aligned}
\end{equation}
\begin{equation}
   \frac{\partial }{\partial x} \int_{0}^{k} E \dif k \approx \frac{\partial }{\partial t} \int_{0}^{\infty} E \dif k  
\end{equation}
能流
\begin{equation}
   - S \equiv - \int_{k}^{\infty} T \dif k 
\end{equation}
\begin{equation}
   \Pi \equiv \int_{k}^{\infty} E \dif k 
\end{equation}
封闭方程
\begin{equation}
   S = 2\nu \int_{k}^{\infty} k^2 E \dif k 
\end{equation}


\subsection{封闭方法}

\subsubsection{Obukhov (1940)}

\begin{equation}
   S = \gamma \int_{k}^{\infty} E \dif k \sqrt{2 \int_{0}^{k} k^2 E \dif k }  
\end{equation}

Millsaps (1955)

解法:令 $2 \int_{0}^{k} k^2 E \dif k = Z^2 \implies k^2 E = Z\cdot Z'$
\begin{equation}
   \gamma \int_{k}^{\infty} E Z = 2 \nu \int_{k}^{\infty} k^2 E \implies \varepsilon - 2 \nu \int_{0}^{k} k^2 E = \gamma Z \int_{k}^{\infty} E \implies\varepsilon - \nu Z^2 = \gamma Z \int_{k}^{\infty} E     
\end{equation}
\begin{equation}
   \frac{\varepsilon}{Z} - \nu Z = g \int_{k}^{\infty} E 
\end{equation}
\begin{equation}
   \left( - \frac{\varepsilon}{Z^2} - \nu \right) Z' = - \gamma E = - \gamma \frac{Z}{k^2}
\end{equation}
\begin{equation}
   \frac{\varepsilon}{Z^2} + \nu = \gamma \frac{Z}{k^2}
\end{equation}
其中
\begin{equation}
   k^2 = \frac{\gamma Z^3}{\varepsilon + \nu Z^2}
\end{equation}
注意到$Z \in \left[0, \sqrt{\frac{\varepsilon}{\nu}}  \right]$ 则 $k^2 \le \frac{\gamma}{2\varepsilon} \left( \frac{\varepsilon}{\nu}^{\frac{3}{2}} \right) \equiv k_{\max}$
在$k\le k_{\max}$ 内估算(当$k$ 很大时)
\begin{equation}
   k^2 \approx \frac{\gamma}{2\varepsilon} Z^3
\end{equation}
\begin{equation}
   Z \approx \left( \frac{2\varepsilon}{\nu} \right)^{\frac{1}{3}} k^{\frac{2}{3}}
\end{equation}
\begin{equation}
   E = \frac{Z Z'}{k^2} = \left( \frac{2\varepsilon}{\nu} \right) ^{\frac{2}{3}} k^{- \frac{5}{3}}
\end{equation}


\subsubsection{Heisonberg--Weizsacker 模型}

\begin{equation}
   S = 2\gamma \int_{k}^{\infty} \left( k^{-3} E \right) ^{\frac{1}{2}}\dif k \int_{0}^{k} k^2 E \dif k  
\end{equation}

$k<k_d$ : $E \approx \left( \frac{8\varepsilon}{9\gamma} \right)^{\frac{2}{3}} k^{-\frac{5}{7}} $

$k\ge k_d: E \approx \left( \frac{\gamma\varepsilon}{2\nu^2} \right) ^2 k^{-7}$

Ellison 修正
\begin{equation}
   S = \alpha k E \sqrt{2 \int_{0}^{k} k^2 E } 
\end{equation}

Hinge 修正
\begin{equation}
   S = \alpha \int_{0}^{k}  \sqrt{kE} \int_{k}^{\infty} E  
\end{equation}


\subsubsection{Karman 模型 (1948)}

\begin{equation}
   S = 2\gamma \int_{k}^{\infty} E^{\alpha} k^{\beta} \int_{0}^{k} E^{\frac{3}{2} - \alpha} k^{\frac{1}{2} - \beta}  
\end{equation}


\subsubsection{Kovasznay 模型}

\begin{equation}
   S = \beta k^{\frac{5}{2}} E^{\frac{3}{2}}
\end{equation}
\begin{equation}
   \begin{cases}
	  E \sim k^{-\frac{5}{3}}, & k<k_d\\
	  E = 0, & k\ge k_d
   \end{cases}
\end{equation}


\subsubsection{Pao YH 模型(1965)}

\begin{equation}
   S = \sigma(k) E(k) = \alpha ^{-1} \varepsilon^{\frac{1}{3}} k^{\frac{5}{3}} E(k)
\end{equation}
\begin{equation}
   \alpha ^{-1} \varepsilon^{\frac{1}{3}} k^{\frac{5}{3}} E = 2\nu \int_{k}^{\infty} k^2 E 
\end{equation}
\begin{equation}
   \alpha ^{-1} \varepsilon^{\frac{1}{3}} \left[ \frac{5}{3} k^{\frac{2}{3}}E + k^{\frac{5}{3}} \frac{\dif E}{\dif k} \right] = - 2 \nu k^2 E
\end{equation}
\begin{equation}
   \frac{1}{E} \frac{\dif E}{\dif k} = -2 \nu \alpha \varepsilon^{-\frac{1}{3}} k^{\frac{1}{3}} - \frac{5}{3} k^{-1}
\end{equation}
\begin{equation}
   E = C_k \varepsilon^{\frac{1}{3}} k^{-\frac{5}{3}} \me^{-\frac{3}{2}\alpha \nu \varepsilon^{-\frac{1}{3}} k^{\frac{4}{3}}}
\end{equation}
若$\bm{v} $ 解析, $k\to \infty, E \sim \me^{-\beta k}$

\section{HIT衰变早期的相似性解}

考虑相似性解: 引入$v,l$ 都只是$t$ 的函数, $E = V^2 l F(x),\, x = kl$, $T = V^3 W(x)$
\begin{equation}
   \frac{\partial E}{\partial t} = \frac{\dif }{\dif t} \left( V^2 l \right) F(x) + V^2 l F' k \frac{\dif l}{\dif t} = V^3 W - 2 \nu k^2 V^2 l F
\end{equation}
\begin{equation}
   V^2 \frac{\dif l}{\dif t} x F' + \left[ \frac{\dif }{\dif t} (V^2l) + 2\nu V^2 l^{-1} x^2 \right] F - V^3 W = 0
\end{equation}
同除$V^3$
\begin{equation}
   \frac{1}{V} \frac{\dif l}{\dif t} x F' + \left[ \frac{1}{V^3}\frac{\dif }{\dif t} (V^2l) + 2\nu \frac{1}{lV} l^{-1} x^2 \right] F - W = 0
\end{equation}
要求: $\frac{1}{V} \frac{\dif l}{\dif t} = \alpha$
\begin{equation}
   \frac{1}{V^3} \frac{\dif }{\dif t}(V^2l) = \frac{1}{V} \frac{\dif l}{\dif t} + \frac{2l}{V^2} \frac{\dif V}{\dif t}, \quad \frac{\nu}{lV} = \alpha_3
\end{equation}
则
\begin{equation}
   \alpha_1 x F' + \left[ \alpha_1 + \alpha_2 + \alpha_3 x^2 \right] F - W = 0
\end{equation}
\begin{equation}
   \frac{1}{V} \frac{\dif l}{\dif t} = \alpha_1 \implies 2 V \frac{\dif l}{\dif t} = 2V^2\alpha_1
\end{equation}
\begin{equation}
   \frac{2l}{V^2} \frac{\dif V}{\dif t} = \alpha_2 \implies 2l \frac{\dif V}{\dif t} = V^2 \alpha_2
\end{equation}
\begin{equation}
   2 \frac{\dif (lV)}{\dif t} = V^2(2\alpha_1+\alpha_2) \implies 2\alpha_1+\alpha_2 = 0
\end{equation}
\begin{equation}
   \frac{1}{V} \frac{\dif }{\dif t} \left( \frac{\frac{\nu}{\alpha_3}}{V} \right)  = \alpha_1
\end{equation}
\begin{equation}
   \frac{1}{V} \frac{\dif }{\dif t}\left( \frac{1}{V} \right) = \frac{\alpha_1 \alpha_3}{\nu}
\end{equation}
\begin{equation}
   \frac{\dif }{\dif t} \left( \frac{1}{V} \right) ^2 = \frac{2 \alpha_1 \alpha_3}{\nu}
\end{equation}
\begin{equation}
   \left( \frac{1}{V} \right) ^2 = \frac{2\alpha_1\alpha_3}{\nu}t + C
\end{equation}
\begin{equation}
   l^2 = \left( \frac{\nu}{\alpha_3 V} \right) ^2 = \frac{\nu^2}{\alpha_3^2} \left[ \frac{2\alpha_1\alpha_3}{\nu}t + C \right]  = \frac{2\nu \alpha_1}{\alpha_3}t + C'
\end{equation}
\begin{equation}
   \frac{3}{2} \left< u^2 \right> = \int_{0}^{\infty} E \dif k = V^2 \int_{0}^{\infty} F(x) \dif x \sim V^2  \sim (t+C)^{-1}
\end{equation}
\begin{equation}
   \frac{3}{2} \frac{\dif }{\dif t} \left< u^2 \right> = - \varepsilon \sim \left( t + C \right) ^{-2}
\end{equation}
\begin{equation}
   \frac{\dif \frac{3}{2} \left< u^2 \right> }{\dif t} \frac{1}{\frac{3}{2}\left< u^2 \right> } = -(t+C)^{-1} = \frac{-15\nu \frac{\left< u^2 \right> }{\lambda^2}}{\frac{3}{2}\left< u^2 \right> } = -10 \frac{\nu}{\lambda^2}
\end{equation}
\begin{equation}
   R_0 \equiv \frac{\sqrt{\left< u^2 \right> \mid_{t=0}} \lambda_0}{\nu}
\end{equation}
\begin{equation}
   \int_{0}^{\infty} 2 x^2 F(x) \dif x = \frac{\varepsilon t^2}{\nu} = \frac{3}{20} R_0^2 \equiv R 
\end{equation}

Landau 质疑(1944)

\begin{equation}
   S_2 (r) = C_2 \varepsilon^{\frac{2}{3}} r^{\frac{2}{3}}, \quad \eta \ll r \le  L
\end{equation}
$C_2$ 不唯一.

K62理论

$v \sim \varepsilon r \implies S_2(r) \propto (\varepsilon_r r)^{\frac{2}{3}}$ 粗粒化的耗散率 $\varepsilon_r \equiv \frac{1}{V} \int_{B(r)}^{} \varepsilon \dif V $

设$\ln \varepsilon_r$ 为正态分布
\begin{equation}
   \left< \ln \varepsilon \right>  = a(r) = c \ln \frac{r_0}{r} + a_0
\end{equation}
\begin{equation}
   \left< \left( \ln \varepsilon_r - a \right) ^2 \right> = \sigma(r) = \mu \ln \frac{r_0}{r} + A
\end{equation}
\begin{equation}
   p(\varepsilon_r) = \frac{1}{\sqrt{2\pi} \sigma} \frac{1}{\varepsilon_r} \me^{-\frac{\left( \ln \varepsilon_r - a \right) ^2}{2\sigma^2}}
\end{equation}
\begin{equation}
   \int_{0}^{\infty} x^{n} p(x) \dif x = \int_{0}^{\infty} x^n \frac{1}{\sqrt{2\pi} \sigma} \frac{1}{x} \me^{-\frac{\left( \ln x - a \right) ^2}{2\sigma^2}} \dif x = \me^{na + \frac{\sigma^2n^2}{2}}
\end{equation}
\begin{equation}
   \left< \varepsilon_r^{p} \right> = \left( \me^{a} \right) ^{n} \left( \me^{\frac{\sigma^2}{2}} \right) ^{n^2} \propto \left( \frac{r_0}{r} \right)^{cn} \left( \frac{r_0}{r} \right)^{\frac{\mu}{2}n^2} = \left( \frac{r}{r_0} \right) ^{-cn - \frac{\mu}{2}n^2}
\end{equation}
\begin{equation}
   \xi_p = \frac{p}{3} + \frac{\mu}{2} \left( \frac{p}{3} - \frac{p^2}{9} \right) 
\end{equation}

\begin{equation}
   \left< \varepsilon_{l}^{p} \right> \sim l^{\tau (p)}
\end{equation}
\begin{equation}
   \delta u_{l} \sim \left( \varepsilon_l l \right) ^{\frac{1}{3}}
\end{equation}
\begin{equation}
   \xi(p) = \frac{p}{3} + \tau\left(\frac{p}{3}\right)
\end{equation}
$P(\delta u_l)$ 与$P(\varepsilon_l)$ 有简单的关系
\begin{equation}
   \tau \left( \frac{p}{3} \right) = \frac{\mu}{18} p (3-p)
\end{equation}
\begin{equation}
   \tau \left( p \right) = \frac{\mu}{2} p (1-p)
\end{equation}
$\xi(p)<0$?
\begin{equation}
   S_p (l) = \left( \frac{l}{l_0} \right) ^{\xi(p)} S_p (l_0), \quad \frac{l}{l_0} \le 1
\end{equation}
若$\delta u_l$ 有最大值$U_{\max}$,若某个$\xi_p < 0, l \to 0, S_p + \left< |\delta u_l |^p \right> \le  (2U_{\max})^p$

$\xi_p$ 是凸的 $\xi''_p < 0$ 
\begin{equation}
   \ln S_p (l) = \xi(p) \ln \left( \frac{l}{l_0} \right) + \ln S_p(l_0)
\end{equation}
对$p$ 求导
\begin{equation}
   \frac{S_p'}{S_p} = \xi'_p \ln \left( \frac{l}{l_0} \right) + \frac{S_p'(l_0)}{S_p(l_0)}
\end{equation}
再求导
\begin{equation}
   \frac{S_p'' S_p - (S_p')^2}{S_p^2} = \xi''_p \ln \left( \frac{l}{l_0} \right) + \frac{S_p''(l_0) S_p(l_0) - [S_p'(l_0)]^2}{S_p(l_0)^2}
\end{equation}
由施瓦尔兹不等式得结论,最左边小于0,最右边大于0
\begin{equation}
   S_p = \int_{0}^{\infty} x^{p} P_l (x) \dif x 
\end{equation}
\begin{equation}
   S_p' = \int_{0}^{\infty} x^{p} \ln x P_l (x) \dif x 
\end{equation}
\begin{equation}
   S_p'' = \int_{0}^{\infty} x^{p} \ln^2 x P_l (x) \dif x 
\end{equation}
分布意义下
\begin{equation}
   \delta u_l \equiv W_{ll_0} \delta u_{l_0},\quad l\le l_0,
\end{equation}
其中$W_{ll_0}$ 是随机乘子
\begin{equation}
   \begin{aligned}
	  \varepsilon_l & \equiv W_{ll_1}\varepsilon_{l_1}\\
					&= W_{ll_1} W_{l_1l_2}\varepsilon_{l_2} \\
					&= W_{ll_1} W_{l_1 l_2}W_{l_1 l_2}\varepsilon_{l_2} \\
   \end{aligned}
\end{equation}
\begin{equation}
   \ln W_{ll_1} = \sum_{i=1}^{n} \ln W_{l_i l_{i+1}}
\end{equation}
$W_{ll_1}$ 的分布只与$\frac{l}{l_1}$ 有关。
\begin{equation}
   \begin{aligned}
	  \left< \varepsilon_l^p \right> & = \left< W_{ll_1}^{p} \varepsilon_{l_1}^{p} \right> \\
									 &= \left< W_{ll_1^{p}} \right> \left< \varepsilon_{l_1}^{p} \right>  \\
   \end{aligned}
\end{equation}
\begin{equation}
   \left< W_{ll_1}^{p} \right> = \left( \frac{l}{l_1} \right) ^{\tau(p)}
\end{equation}
\begin{equation}
   \frac{l}{l_1} = \frac{l}{l_2} \frac{l_2}{l_3} \cdots \frac{l_{n_1}}{l_1} = \left( \frac{l}{l_2} \right) ^{n-1}
\end{equation}

分形几何方法,$\beta$-模型,大涡$l_0$小涡$\lambda l_0,\lambda<1$体积$\beta l_0^3 \neq l^3$ $n$次 $l = \lambda^{n}l_0$, $V = \beta^{n} l_0^{3}$ 小涡占有体积比率

个数 $N = \frac{\beta^{n}l_0^{3}}{l^{3}} = \lambda^{n \frac{\ln \beta}{\ln \lambda}}$


\section{数值模拟}

三维 $\bm{u} (\bm{x} ,t)$ , 周期边界条件。

\begin{equation}
   \bm{u} (\bm{x} , t) = \sum_{\bm{k} }^{} \hat{\bm{u} } (\bm{k} ,t) \me^{\mi \bm{k} \cdot \bm{x} },
\end{equation}
\begin{equation}
   \nabla \cdot \bm{u} = 0 \implies \sum_{k}^{} \mi \bm{k} \cdot \hat{\bm{u} } (\bm{k} ,t) \me^{\mi \bm{k} \cdot \bm{x} } = 0 \implies \bm{k} \cdot \hat{\bm{u} }(\bm{k} ,t)=0, 
\end{equation}
\begin{equation}
   \frac{\partial \bm{u} }{\partial t}  + \nabla \cdot (\bm{u} \bm{u} ) = - \nabla \left( \frac{p}{\rho} \right) + \nu \nabla ^2 \bm{u} + \bm{f}.
\end{equation}
亥姆霍兹分解 $\bm{a} = \nabla \phi + \bm{a}_{\bot}$
\begin{equation}
   -k^2 \hat{\phi} = \mi \bm{k}  \cdot \hat{\bm{a} }(\bm{k} ),\quad \hat{\phi} = - \frac{\mi \bm{k} }{k^2} \cdot \hat{\bm{a} }(\bm{k} )
\end{equation}
\begin{equation}
   \hat{\bm{a} }_{\bot} = \hat{\bm{a} } - \mi \bm{k} \hat{\phi} = \hat{\bm{a} } + \mi \bm{k} \frac{\mi \bm{k} }{k^2}\cdot \hat{\bm{a} } = \hat{\bm{a} } - \frac{\bm{k} \bm{k} }{k^2}\cdot \hat{\bm{a} } = \left( I - \frac{\bm{k} \bm{k} }{k^2} \right) \cdot \hat{\bm{a} } = P \cdot \hat{\bm{a} }
\end{equation}
其中
\begin{equation}
   P_{ij} = \delta_{ij} - \frac{k_i k_j}{k^2}
\end{equation}
ODE:
\begin{equation}
   \left( \frac{\dif }{\dif t} + \nu k^2 \right) \hat{u}_i (\bm{k} ,t) = - \mi k_m P_{ij} (\bm{k} ) \cdot \sum_{\bm{p} + \bm{q} = \bm{k} }^{} u_j (\bm{p} ,t) u_m(\bm{q} ,t) + \hat{f}(\bm{k} ,t)
\end{equation}

\begin{equation}
   \sum_{k-\frac{1}{2}\le \abs{\bm{k} }\le k + \frac{1}{2}}^{} \left< \abs{ \hat{u}_i \hat{u}_i } \right> = E(k)
\end{equation}
Kolmogorov 尺度 $\frac{L}{N} \sim \eta$ 
\begin{equation}
   \eta \equiv \left( \frac{\nu^{3}}{\varepsilon} \right)^{\frac{1}{4}}
\end{equation}

螺旋波分解
\begin{equation}
   \left\{
	  \begin{aligned}
		 u & = \sin z \\
		 v & = \cos z \\
		 w & = 0.
	  \end{aligned}
   \right.
\end{equation}
\begin{equation}
   \bm{\omega}  = \bm{u} 
\end{equation}

\begin{equation}
   \left\{
	  \begin{aligned}
		 u & = -\sin z \\
		 v & = \cos z \\
		 w & = 0.
	  \end{aligned}
   \right.
\end{equation}
\begin{equation}
   \bm{\omega}  = - \bm{u} 
\end{equation}


\begin{equation}
   \left\{
	  \begin{aligned}
		 u & = \sin kz \\
		 v & = \cos kz \\
		 w & = 0.
	  \end{aligned}
   \right. \implies
   \begin{aligned}
	  \bm{V}_1^{+} = ( \sin kz, \cos kz, 0) \\
	  \bm{V}_1^{-} = ( -\sin kz, \cos kz, 0)
   \end{aligned}
\end{equation}
\begin{equation}
   \begin{aligned}
	  \bm{V}_2^{+} = ( -\sin kz, \cos kz, 0) \\
	  \bm{V}_2^{-} = ( -\sin kz, - \cos kz, 0)
   \end{aligned}
\end{equation}
$kz \to  \bm{k} \cdot \bm{x} $ 
\begin{equation}
   \left\{
	  \begin{aligned}
		 \bm{V}^{+} = \bm{V}^{+}_1 + \mi \bm{V}^{+}_2\\
		 \bm{V}^{-} = \bm{V}^{-}_1 + \mi \bm{V}^{-}_2
	  \end{aligned}
   \right.
\end{equation}
\begin{equation}
   \bm{u} = \sum_{\bm{k} } c_k \bm{\hat{V}} ^{\pm} (\bm{k} ,t)
\end{equation}
\begin{equation}
   \frac{\dif c_k}{\dif t} = \sum_{m,n} Q_{kmn} c_m c_n + \nu k^2 c_k + f_k
\end{equation}
三波$\bm{m} + \bm{n} + \bm{k} = 0$.

吉田善章1990
\begin{equation}
   \left\{
	  \begin{aligned}
		 &\nabla \times \bm{B}_n = \lambda_n \bm{B} _n \\
		 &\bm{n} \cdot \bm{B}_n = 0 \\
		 &\bm{n} \cdot \nabla \times \bm{B} _n = 0
	  \end{aligned}
   \right.
\end{equation}

\subsection{湍流数值模拟}

Reynold 应力, $R_{ij} \equiv \left< u_i' u_j' \right> $ 湍流模型.

二阶矩方法
\begin{equation}
   \frac{\partial \left< u_i' u_j' \right> }{\partial t} = P + D_1 + D_2
\end{equation}
右端分别是产生,扩散和耗散。

涡粘性方法
\begin{equation}
   \left< u_i' u_j' \right>  = 2\nu_T S_{ij} + \frac{2}{3} \delta_{ij} k ,\quad k \equiv - \frac{1}{2} \left< u_i' u_j' \right> 
\end{equation}
\begin{equation}
   \frac{\dif \nu_T}{\dif t} = \cdots
\end{equation}

Calay--Hamilton 定理


















% \nocite{*}

% % \newpage
\bibliographystyle{plain}
% \clearpage
\phantomsection

\addcontentsline{toc}{section}{参考文献} %向目录中添加条目,以章的名义
\bibliography{homework}

\end{document}
